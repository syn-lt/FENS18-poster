  
\section{Network simulations}
\vspace{-0.2cm}

What parameters of a cortical circuit could match $\mu_b$ in the stochastic model and thus similarly shape the lifetime and weight distributions of synapses? To test this we simulated a detailed network model similar to \cite{Miner2016}. In the model we simulated $N_e = 400$ excitatory and $N_i = 80$ inhibitory leaky integrate-and-fire neurons with membrane noise over $T=\text{\SI{1000}{s}}$. Synapses are conductance based and network connections within the excitatory population are inserted randomly over time at a low weight and eliminated if the weight falls below a global threshold. Spike-timing dependent plasticity and slow multiplicative synaptic normalization drive the synapse dynamics. We found that a parameter $D$, defined as the ratio between potentiation and depression through spike-timing dependent plasticity in the network,
%
\begin{align}
  D = \frac{|\text{synaptic potentiation}|}{|\text{synaptic depression}|},
\end{align}
%
shows a similar shaping of lifetime distributions in the network model (Fig.~\ref{fig:netwsim}A). We see however qualitatively different behaviour in the dynamics of the synaptic weight distributions (Fig.~\ref{fig:netwsim}B).

\vspace{0.7cm}
\begin{overpic}[width=\columnwidth]%
  % 110, 133
  {figures/lifetimes_single.pdf}
  %\put(32,\ylin){anisotropic}
  \put(1,28){\normalfont \textbf{A}}
  \put(52,28){\normalfont \textbf{B}}
\end{overpic}
\captionof{figure}{In detailed network simulations the synaptic lifetime and weight distributions are shaped by the overall ratio $D$ of synaptic potentiation and depression. \label{fig:netwsim}}
\vspace{2.3cm}

Interestingly, we find that a large number of other network parameters, such as for example membrane noise, do not affect the synaptic lifetimes. This points towards a fundamental relationship between the additive weight dynamics and the lifetimes of synapses.






